The Large Hadron Collider (LHC) is the largest and most powerful particle collider in the world.
It is constructed by the European Organization for Nuclear Research (CERN) between 1998 and 2008 in collaboration with 10 000 scientists and engineers from over 100 countries and 1000 research institutes.
The LHC provides proton beams circulating in oppositie directions and colliding at high energy.
Those proton-proton collisions happen at specific interation points, where particle detectors are situated.
The Compact Muon Solenoid detector (CMS) is one of these detectos.
It is built as the joint efforts of nearly 4000 individuals from over 40 countries and 200 research institutes.
The physical quantities of particles emitting from the proton-proton collisions can be well measured by the CMS detector.
