The \ttbar production events are simulated using Monte Carlo (MC) programs.
The \ttbar is simulated using quantum chromodynamics (QCD) at next-to-leading-order (NLO) precision through the matrix element (ME) in the \POWHEG 2.0 event generator~\cite{Sim:powheg1, Sim:powheg2, Sim:powheg3, Sim:powheg4}.
The value of the top quark mass (\Mt) is set to 172.5\GeV.
The \POWHEG output is combined with the parton shower (PS) simulation of \PYTHIA 8.205~\cite{Sim:pythia2}, with the underlying-event (UE) tune CP5~\cite{Sim:CP5}.
The parton distribution functions (PDFs) NNPDF 3.1~\cite{Sim:NNPDF3.1} at next-to-NLO order (NNLO) are used to model the data.
Samples of \ttbar with different values of CEDM are simulated with the \MADGRAPH generator~\cite{Sim:madgraph} at leading-order (LO) precision interfaced with \PYTHIA.
These samples are used as a cross-check of the model dependency of each CP observable.

Several backgrounds are considered with single top quark production being the leading contribution.
This is simulated at NLO using \MADGRAPH 2.4.2 with the FxFx matching scheme~\cite{Sim:FXFX} for $s$-channel production and \POWHEG for $t$-channel and $\PQt\PW$ production.
All samples are interfaced with \PYTHIA through the UE tune CUETP8M1~\cite{Sim:CUETP8M1} for $t$-channel production for the 2016 data and CP5 for the rest.
Diboson ($\PV\PV$), \Wjets, Drell--Yan (DY), and QCD multijet productions are simulated at LO using the \MADGRAPH 2.4.4 generator.
They are then interfaced with \PYTHIA using the MLM matching scheme~\cite{Sim:MLMmatching}.
Events coming from the decay of \ttbar into either dilepton+jets or hadronic multijets can be incorrectly included in the signal event sample due to particle misidentification.
They are considered as a background.
The background from \PW boson events with heavy-flavor quarks (\WHF), which is not a large background, is important during the estimation of the systematic uncertainties, and will be discussed in detail in Section~\ref{sec:uncertainty}.
List of all the simulated samples with various hard process used in this analysis can be found in Tables~\ref{tab:mc_list_16},~\ref{tab:mc_list_17}, and~\ref{tab:mc_list_18}.

The simulation of the experimental apparatus is based on \GEANTfour~\cite{Sim:geant4}.
To model the effect of additional \pp interactions within the same or nearby bunch crossings (pileup), simulated minimum-bias interactions are included in the simulated samples~\cite{Sim:pileup}.
The number of pileup interactions in the simulation is reweighted to match the distribution in data.
\begin{table}[H]
    \caption{List of 2016 simulated samples used in this analysis}
    \label{tab:mc_list_16}
    \centering
    \resizebox{\textwidth}{!}{
        \begin{tabular}{|l|cc|c|}
    \hline
    Process & Cross section ($pb$) & K Factor\\
    \hline
    \multicolumn{3}{|c|}{Signal}\\
    \hline
    TTToSemiLeptonic\_TuneCP5\_PSweights\_13TeV-powheg-pythia8                                   &  $831.76 * 0.44113$(NNLO)       & 1 \\
    TTTo2L2Nu\_TuneCP5\_PSweights\_13TeV-powheg-pythia8                                          &  $831.76 * 0.10706$(NNLO)       & 1 \\
    TTToHadronic\_TuneCP5\_PSweights\_13TeV-powheg-pythia8                                       &  $831.76 * 0.45441$(NNLO)       & 1 \\
    \hline
    \multicolumn{3}{|c|}{Background}\\
    \hline 
    QCD\_HT100to200\_TuneCUETP8M1\_13TeV-madgraphMLM-pythia8                                     &     $27990000$(LO)              & 1 \\
    QCD\_HT200to300\_TuneCUETP8M1\_13TeV-madgraphMLM-pythia8                                     &     $1712000$(LO)               & 1 \\
    QCD\_HT300to500\_TuneCUETP8M1\_13TeV-madgraphMLM-pythia8                                     &     $347700$(LO)                & 1 \\
    QCD\_HT500to700\_TuneCUETP8M1\_13TeV-madgraphMLM-pythia8                                     &     $32100$(LO)                 & 1 \\
    QCD\_HT700to1000\_TuneCUETP8M1\_13TeV-madgraphMLM-pythia8                                    &     $6831$(LO)                  & 1 \\
    QCD\_HT1000to1500\_TuneCUETP8M1\_13TeV-madgraphMLM-pythia8                                   &     $1207$(LO)                  & 1 \\
    QCD\_HT1500to2000\_TuneCUETP8M1\_13TeV-madgraphMLM-pythia8                                   &     $119.9$(LO)                 & 1 \\
    QCD\_HT2000toInf\_TuneCUETP8M1\_13TeV-madgraphMLM-pythia8                                    &     $25.2$(LO)                  & 1  \\
    \hline
    WW\_TuneCUETP8M1\_13TeV-pythia8                                                              &   $118.7$(NLO)                  &1\\
    ZZ\_TuneCUETP8M1\_13TeV-pythia8                                                              &   $16.5$(NLO)                   &   1  \\      
    WZ\_TuneCUETP8M1\_13TeV-pythia8                                                              &   $47.1$(NLO)                   &   1  \\                
    ST\_s-channel\_4f\_leptonDecays\_TuneCP5\_PSweights\_13TeV-amcatnlo-pythia8                  &   $3.36$(NLO)                     &   1    \\
    ST\_t-channel\_top\_4f\_inclusiveDecays\_13TeV-powhegV2-madspin-pythia8\_TuneCUETP8M1         &   $136.02$(NLO)                   &   1    \\
    ST\_t-channel\_antitop\_4f\_InclusiveDecays\_TuneCP5\_PSweights\_13TeV-powheg-pythia8        &   $80.95$(NLO)                    &   1    \\
    ST\_tW\_top\_5f\_inclusiveDecays\_TuneCP5\_PSweights\_13TeV-powheg-pythia8                   &   $35.6$(NLO)                     &   1    \\
    ST\_tW\_antitop\_5f\_inclusiveDecays\_TuneCP5\_PSweights\_13TeV-powheg-pythia8               &   $35.6$(NLO)                     &   1    \\
    \hline
    WJetsToLNu\_HT-100To200\_TuneCUETP8M1\_13TeV-madgraphMLM-pythia8                             &     $1345.7$(LO)                & 1.21  \\
    WJetsToLNu\_HT-200To400\_TuneCUETP8M1\_13TeV-madgraphMLM-pythia8                             &     $359.7$(LO)                 & 1.21  \\
    WJetsToLNu\_HT-400To600\_TuneCUETP8M1\_13TeV-madgraphMLM-pythia8                             &      $48.9$(LO)                  & 1.21  \\
    WJetsToLNu\_HT-600To800\_TuneCUETP8M1\_13TeV-madgraphMLM-pythia8                             &      $12.1$(LO)                  & 1.21  \\
    WJetsToLNu\_HT-800To1200\_TuneCUETP8M1\_13TeV-madgraphMLM-pythia8                            &     $5.5$(LO)                   & 1.21  \\
    WJetsToLNu\_HT-1200To2500\_TuneCUETP8M1\_13TeV-madgraphMLM-pythia8                           &     $1.3$(LO)                   & 1.21  \\
    WJetsToLNu\_HT-2500ToInf\_TuneCUETP8M1\_13TeV-madgraphMLM-pythia8                            &     $3.2\times 10^{-2}$(LO)     & 1.21  \\
    DYJetsToLL\_M-50\_HT-70to100\_TuneCUETP8M1\_13TeV-madgraphMLM-pythia8             &     $169.9$(LO)                 & 1.23  \\
    DYJetsToLL\_M-50\_HT-100to200\_TuneCUETP8M1\_13TeV-madgraphMLM-pythia8            &     $147.4$(LO)                 & 1.23  \\
    DYJetsToLL\_M-50\_HT-200to400\_TuneCUETP8M1\_13TeV-madgraphMLM-pythia8            &     $41.0$(LO)                  & 1.23  \\
    DYJetsToLL\_M-50\_HT-400to600\_TuneCUETP8M1\_13TeV-madgraphMLM-pythia8            &     $5.7$(LO)                   & 1.23  \\
    DYJetsToLL\_M-50\_HT-600to800\_TuneCUETP8M1\_13TeV-madgraphMLM-pythia8            &     $1.4$(LO)                   & 1.23  \\
    DYJetsToLL\_M-50\_HT-800to1200\_TuneCUETP8M1\_13TeV-madgraphMLM-pythia8           &     $6.3\times 10^{-1}$(LO)     & 1.23  \\
    DYJetsToLL\_M-50\_HT-1200to2500\_TuneCUETP8M1\_13TeV-madgraphMLM-pythia8          &     $1.5\times 10^{-1}$(LO)     & 1.23  \\
    DYJetsToLL\_M-50\_HT-2500toInf\_TuneCUETP8M1\_13TeV-madgraphMLM-pythia8           &     $3.6\times 10^{-3}$(LO)     & 1.23  \\
    \hline
\end{tabular}

    }
\end{table}
\begin{table}[H]
    \caption{List of 2017 simulated samples used in this analysis}
    \label{tab:mc_list_17}
    \centering
    \resizebox{\textwidth}{!}{
        \begin{tabular}{|l|cc|c|}
    \hline
    Process & Cross section ($pb$) & K Factor\\
    \hline
    \multicolumn{3}{|c|}{Signal}\\
    \hline
    TTToSemiLeptonic\_TuneCP5\_13TeV-powheg-pythia8                                              &  $831.76 * 0.44113$(NNLO)       & 1 \\
    TTTo2L2Nu\_TuneCP5\_PSweights\_13TeV-powheg-pythia8                                          &  $831.76 * 0.10706$(NNLO)       & 1 \\
    TTToHadronic\_TuneCP5\_PSweights\_13TeV-powheg-pythia8                                       &  $831.76 * 0.45441$(NNLO)       & 1 \\
    \hline
    TT\_CEDM\_dtG0-MadGraph5-pythia8                                                    &  482.5(LO) & 1 \\
    TT\_CEDM\_dtG1-MadGraph5-pythia8                                                    &  497.6(LO) & 1 \\ 
    TT\_CEDM\_dtG2-MadGraph5-pythia8                                                    &  581.5(LO) & 1 \\ 
    TT\_CEDM\_dtG3-MadGraph5-pythia8                                                    &  715.9(LO) & 1 \\ 
    TT\_CEDM\_dtG4-MadGraph5-pythia8                                                    &  939.6(LO) & 1 \\ 
    TT\_CEDM\_dtG5-MadGraph5-pythia8                                                    &  1196 (LO) & 1 \\ 
    \hline
    \multicolumn{3}{|c|}{Background}\\
    \hline 
    QCD\_HT100to200\_TuneCP5\_13TeV-madgraph-pythia8                                     &     $27990000$(LO)              & 1 \\
    QCD\_HT200to300\_TuneCP5\_13TeV-madgraph-pythia8                                     &     $1712000$(LO)               & 1 \\
    QCD\_HT300to500\_TuneCP5\_13TeV-madgraph-pythia8                                     &     $347700$(LO)                & 1 \\
    QCD\_HT500to700\_TuneCP5\_13TeV-madgraph-pythia8                                     &     $32100$(LO)                 & 1 \\
    QCD\_HT700to1000\_TuneCP5\_13TeV-madgraph-pythia8                                    &     $6831$(LO)                  & 1 \\
    QCD\_HT1000to1500\_TuneCP5\_13TeV-madgraph-pythia8                                   &     $1207$(LO)                  & 1 \\
    QCD\_HT1500to2000\_TuneCP5\_13TeV-madgraph-pythia8                                   &     $119.9$(LO)                 & 1 \\
    QCD\_HT2000toInf\_TuneCP5\_13TeV-madgraph-pythia8                                    &     $25.2$(LO)                  & 1  \\
    \hline
    WW\_TuneCP5\_13TeV-pythia8                                                              &   $118.7$(NLO)                  &   1 \\                                               
    ZZ\_TuneCP5\_13TeV-pythia8                                                              &   $16.5$(NLO)                   &   1  \\      
    WZ\_TuneCP5\_13TeV-pythia8                                                              &   $47.1$(NLO)                   &   1  \\                
    ST\_s-channel\_4f\_leptonDecays\_TuneCP5\_13TeV-amcatnlo-pythia8                        &   $3.36$(NLO)                     &   1    \\
    ST\_t-channel\_top\_4f\_inclusiveDecays\_TuneCP5\_13TeV-powhegV2-madspin-pythia8          &   $136.02$(NLO)                   &   1    \\
    ST\_t-channel\_antitop\_4f\_inclusiveDecays\_TuneCP5\_13TeV-powhegV2-madspin-pythia8      &   $80.95$(NLO)                    &   1    \\
    ST\_tW\_top\_5f\_inclusiveDecays\_TuneCP5\_13TeV-powheg-pythia8                          &   $35.6$(NLO)                     &   1    \\
    ST\_tW\_antitop\_5f\_inclusiveDecays\_TuneCP5\_13TeV-powheg-pythia8                     &   $35.6$(NLO)                     &   1    \\
    \hline
    WJetsToLNu\_HT-100To200\_TuneCP5\_13TeV-madgraphMLM-pythia8                             &     $1345.7$(LO)                & 1.21  \\
    WJetsToLNu\_HT-200To400\_TuneCP5\_13TeV-madgraphMLM-pythia8                             &     $359.7$(LO)                 & 1.21  \\
    WJetsToLNu\_HT-400To600\_TuneCP5\_13TeV-madgraphMLM-pythia8                             &      $48.9$(LO)                  & 1.21  \\
    WJetsToLNu\_HT-600To800\_TuneCP5\_13TeV-madgraphMLM-pythia8                             &      $12.1$(LO)                  & 1.21  \\
    WJetsToLNu\_HT-800To1200\_TuneCP5\_13TeV-madgraphMLM-pythia8                            &     $5.5$(LO)                   & 1.21  \\
    WJetsToLNu\_HT-1200To2500\_TuneCP5\_13TeV-madgraphMLM-pythia8                           &     $1.3$(LO)                   & 1.21  \\
    WJetsToLNu\_HT-2500ToInf\_TuneCP5\_13TeV-madgraphMLM-pythia8                            &     $3.2\times 10^{-2}$(LO)     & 1.21  \\
    DYJetsToLL\_M-50\_HT-70to100\_TuneCP5\_13TeV-madgraphMLM-pythia8             &     $169.9$(LO)                 & 1.23  \\
    DYJetsToLL\_M-50\_HT-100to200\_TuneCP5\_13TeV-madgraphMLM-pythia8            &     $147.4$(LO)                 & 1.23  \\
    DYJetsToLL\_M-50\_HT-200to400\_TuneCP5\_13TeV-madgraphMLM-pythia8            &     $41.0$(LO)                  & 1.23  \\
    DYJetsToLL\_M-50\_HT-400to600\_TuneCP5\_13TeV-madgraphMLM-pythia8            &     $5.7$(LO)                   & 1.23  \\
    DYJetsToLL\_M-50\_HT-600to800\_TuneCP5\_13TeV-madgraphMLM-pythia8            &     $1.4$(LO)                   & 1.23  \\
    DYJetsToLL\_M-50\_HT-800to1200\_TuneCP5\_13TeV-madgraphMLM-pythia8           &     $6.3\times 10^{-1}$(LO)     & 1.23  \\
    DYJetsToLL\_M-50\_HT-1200to2500\_TuneCP5\_13TeV-madgraphMLM-pythia8          &     $1.5\times 10^{-1}$(LO)     & 1.23  \\
    DYJetsToLL\_M-50\_HT-2500toInf\_TuneCP5\_13TeV-madgraphMLM-pythia8           &     $3.6\times 10^{-3}$(LO)     & 1.23  \\
    \hline
\end{tabular}

    }
\end{table}
\begin{table}[H]
    \caption{List of 2018 simulated samples used in this analysis}
    \label{tab:mc_list_18}
    \centering
    \resizebox{\textwidth}{!}{
        \begin{tabular}{|l|cc|c|}
    \hline
    Process & Cross section ($pb$) & K Factor\\
    \hline
    \multicolumn{3}{|c|}{Signal}\\
    \hline
    TTToSemiLeptonic\_TuneCP5\_13TeV-powheg-pythia8                                              &  $831.76 * 0.44113$(NNLO)       & 1 \\
    TTTo2L2Nu\_TuneCP5\_13TeV-powheg-pythia8                                          &  $831.76 * 0.10706$(NNLO)       & 1 \\
    TTToHadronic\_TuneCP5\_13TeV-powheg-pythia8                                       &  $831.76 * 0.45441$(NNLO)       & 1 \\
    \hline
    \multicolumn{3}{|c|}{Background}\\
    \hline 
    QCD\_HT100to200\_TuneCP5\_13TeV-madgraphMLM-pythia8                                     &     $27990000$(LO)              & 1 \\
    QCD\_HT200to300\_TuneCP5\_13TeV-madgraphMLM-pythia8                                     &     $1712000$(LO)               & 1 \\
    QCD\_HT300to500\_TuneCP5\_13TeV-madgraphMLM-pythia8                                     &     $347700$(LO)                & 1 \\
    QCD\_HT500to700\_TuneCP5\_13TeV-madgraphMLM-pythia8                                     &     $32100$(LO)                 & 1 \\
    QCD\_HT700to1000\_TuneCP5\_13TeV-madgraphMLM-pythia8                                    &     $6831$(LO)                  & 1 \\
    QCD\_HT1000to1500\_TuneCP5\_13TeV-madgraphMLM-pythia8                                   &     $1207$(LO)                  & 1 \\
    QCD\_HT1500to2000\_TuneCP5\_13TeV-madgraphMLM-pythia8                                   &     $119.9$(LO)                 & 1 \\
    QCD\_HT2000toInf\_TuneCP5\_13TeV-madgraphMLM-pythia8                                    &     $25.2$(LO)                  & 1  \\
    \hline
    WW\_TuneCP5\_13TeV-pythia8                                                              &   $118.7$(NLO)                  &   1 \\                                               
    ZZ\_TuneCP5\_13TeV-pythia8                                                              &   $16.5$(NLO)                   &   1  \\      
    WZ\_TuneCP5\_13TeV-pythia8                                                              &   $47.1$(NLO)                   &   1  \\                
    ST\_s-channel\_4f\_leptonDecays\_TuneCP5\_13TeV-madgraph-pythia8                        &   $3.36$(NLO)                     &   1    \\
    ST\_t-channel\_top\_4f\_InclusiveDecays\_TuneCP5\_13TeV-powheg-madspin-pythia8          &   $136.02$(NLO)                   &   1    \\
    ST\_t-channel\_antitop\_4f\_InclusiveDecays\_TuneCP5\_13TeV-powheg-madspin-pythia8      &   $80.95$(NLO)                    &   1    \\
    ST\_tW\_top\_5f\_inclusiveDecays\_TuneCP5\_13TeV-powheg-pythia8                          &   $35.6$(NLO)                     &   1    \\
    ST\_tW\_antitop\_5f\_inclusiveDecays\_TuneCP5\_13TeV-powheg-pythia8                     &   $35.6$(NLO)                     &   1    \\
    \hline
    WJetsToLNu\_HT-100To200\_TuneCP5\_13TeV-madgraphMLM-pythia8                             &     $1345.7$(LO)                & 1.21  \\
    WJetsToLNu\_HT-200To400\_TuneCP5\_13TeV-madgraphMLM-pythia8                             &     $359.7$(LO)                 & 1.21  \\
    WJetsToLNu\_HT-400To600\_TuneCP5\_13TeV-madgraphMLM-pythia8                             &      $48.9$(LO)                  & 1.21  \\
    WJetsToLNu\_HT-600To800\_TuneCP5\_13TeV-madgraphMLM-pythia8                             &      $12.1$(LO)                  & 1.21  \\
    WJetsToLNu\_HT-800To1200\_TuneCP5\_13TeV-madgraphMLM-pythia8                            &     $5.5$(LO)                   & 1.21  \\
    WJetsToLNu\_HT-1200To2500\_TuneCP5\_13TeV-madgraphMLM-pythia8                           &     $1.3$(LO)                   & 1.21  \\
    WJetsToLNu\_HT-2500ToInf\_TuneCP5\_13TeV-madgraphMLM-pythia8                            &     $3.2\times 10^{-2}$(LO)     & 1.21  \\
    DYJetsToLL\_M-50\_HT-70to100\_TuneCP5\_PSweights\_13TeV-madgraphMLM-pythia8             &     $169.9$(LO)                 & 1.23  \\
    DYJetsToLL\_M-50\_HT-100to200\_TuneCP5\_PSweights\_13TeV-madgraphMLM-pythia8            &     $147.4$(LO)                 & 1.23  \\
    DYJetsToLL\_M-50\_HT-200to400\_TuneCP5\_PSweights\_13TeV-madgraphMLM-pythia8            &     $41.0$(LO)                  & 1.23  \\
    DYJetsToLL\_M-50\_HT-400to600\_TuneCP5\_PSweights\_13TeV-madgraphMLM-pythia8            &     $5.7$(LO)                   & 1.23  \\
    DYJetsToLL\_M-50\_HT-600to800\_TuneCP5\_PSweights\_13TeV-madgraphMLM-pythia8            &     $1.4$(LO)                   & 1.23  \\
    DYJetsToLL\_M-50\_HT-800to1200\_TuneCP5\_PSweights\_13TeV-madgraphMLM-pythia8           &     $6.3\times 10^{-1}$(LO)     & 1.23  \\
    DYJetsToLL\_M-50\_HT-1200to2500\_TuneCP5\_PSweights\_13TeV-madgraphMLM-pythia8          &     $1.5\times 10^{-1}$(LO)     & 1.23  \\
    DYJetsToLL\_M-50\_HT-2500toInf\_TuneCP5\_PSweights\_13TeV-madgraphMLM-pythia8           &     $3.6\times 10^{-3}$(LO)     & 1.23  \\
    \hline
\end{tabular}

    }
\end{table}
